\documentclass[a4]{article}
%\usepackage{czech}
\usepackage{graphicx}
\usepackage[utf8]{inputenc}   % pro unicode UTF-8
\usepackage{booktabs}
\usepackage{hyperref}
%#\usepackage{qtree}
%\usepackage{graphviz}
\usepackage{authblk}
%\usepackage{tikz-dependency}
\usepackage[ampersand]{easylist}

%\usepackage{verse}

\def\furl#1{\footnote{\url{#1}}}


\begin{document}

\title{Neuronový strojový překlad z němčiny do češtiny}

\author{Dominik Macháček}
\affil{
%Saarland University, Saarbrücken, Germany,
dominik.machacek@matfyz.cz
}

\date{\today}

%\pacs{PACS numbers go here. These are classification codes for your  research. See {\tt http://publish.aps.org/PACS/} for more info.}
\maketitle

%\begin{abstract}
%TODO: An abstract is a great convenience for the reader and is required by all journals.
%\end{abstract}


\section{Úvod}

V tomto článku shrnujeme experimenty trénování neuronového strojového
překladu z~němčiny do češtiny.

\section{Jiné strojové překlady}

\subsection{Zeman: frázový překlad pro WMT13}

Dosud nejlepší strojový překlad z~němčiny do češtiny vytvořil Dan Zeman
v~roce 2014. Systém je popsán v~článku \cite{zeman}.

Jedná se o frázový překlad s~použitím systému Moses. Pro trénování autor
použil paralelní korpusy Europarl a News Commentary ve verzích, jaké byly
k~dispozici jako trénovací data pro
WMT13\furl{http://www.statmt.org/wmt13/translation-task.html}. Pro vývoj
sloužila sada překladů novinových článků News Commentary 2011, pro testování
News Commentary 2013.

Ve fázi preprocessingu byla data tokenizována, zvláštní znaky byly
normalizovány nebo odstraněny. Následně byly normalizována reprezentace
uvozovek a byl aplikován truecasing.

%Například slovo {\it Nationalessen} je
%složeno ze slov {\it National} (národní) a {\it Essen} (jídlo). Samo o sobě
%ve slovníku být nemusí, ale každé z~jeho kořenů ano. Proto 
%

Dále se autor snažil vyřešit problém s~německými složenými slovy, která
zvyšují počet neznámých slov. K tomu použil nástroj
Morfessor\cite{morfessor}. Text je před překládáním rozdělen na morfémy,
poté přeložen a poté opět složen do celých slov.

TODO: jak to vlastně je? A co lemmatizace?

Pro přehledné porovnání s~ostatními překladatelskými systémy je výsledek
nahrán na \url{matrix.statmt.org}.

\subsection{Tlustý: Využí hrubé reprezentace slov ve strojovém
překladu do češtiny}

V roce 2016 vznikla diplomová práce Marka Tlustého \cite{tlusty}, ve které se autor
zabývá dělením německých a maďarských slov na kratší části podle jejich
morfologického významu. Dělí se tedy např. na morfémy nebo jednotlivé kořeny
složených slov. To má sloužit ke snížení podílu neznámých slov v~testovací
množině oproti trénovací a ke zlepšení frázového překladu. 

Autor experimentoval s~několika strategiemi a existujícími nástroji, téměř
všechny vykázaly zlepšení podle metrik BLEU a METEOR oproti systémům bez
hrubé reprezentace.  Překlad byl realizován sysémem Moses. Autor použil
korpus Europarl\cite{europarl}, který sám rozdělil na trénovací, testovací a vývojovou
část. 

V~práci není srovnání s~předchozím překladem Dana Zemana, ani v~ní nejsou
vloženy výsledné natrénované modely. Rovněž práce neobsahuje podrobný manuál, jak
zreplikovat její nejlepší výsledky.

Neuronový strojový překlad se v~práci vůbec nezmiňuje. 
U něj se rovněž používá hrubá reprezentace slov, nejčastěji
tzv. byte pair encoding, u něhož se dělí slova na podslova podle absolutní četnosti jejich
částí. Na morfologický význam vzniklých podslov se nebere zřetel. Toto je
jiná strategie než u frázového překladu a v~práci se neuvádí. 

%To, že strategie pro hrubou reprezentaci slov pro frázový překlad budou vykazovat dobré
%výsledky i pro neuronový překlad, není zřejmé.


\section{Data a preprocessing}

\subsection{Pročištění europarlu}

\subsection{BPE}

\subsection{Opensubtitles}

\section{Evaluace podle BLEU}

\section{Analýza nejlepšího překladu}

\section{Závěr}


\begin{thebibliography}{99}



\bibitem{zeman}
 Ondřej Bojar, Daniel Zeman: {\sl Czech Machine Translation in the project
 CzechMATE}. Publikováno v~{\sl The Prague Bulletin of Mathematical
 Linguistics}, číslo 101., duben 2014, strany 71–96.

\bibitem{morfessor}
	Mathias Creutz, Krista Lagus: {\sl Unsupervised models for morpheme
	segmentation and
	morphology learning}. ACM Transactions on Speech and Language Processing
	(TLSP), 4(1)(3),
	2007. URL \url{http://dl.acm.org/citation.cfm?id=1187418}.

\bibitem{tlusty}
 Bc. Marek Tlustý: {\sl  Využí hrubé reprezentace slov ve strojovém
překladu do češtiny}. Diplomová práce. Praha, 2016.
% \url{https://github.com/Gldkslfmsd/sekacek}





\end{thebibliography}





\end{document}


